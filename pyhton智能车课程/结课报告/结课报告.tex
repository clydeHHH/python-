\documentclass[UTF8]{ctexart}
\usepackage[]{xeCJK}
\newcommand{\sanhao}{\fontsize{16pt}{24pt}\selectfont}
\newcommand{\xiaosan}{\fontsize{15pt}{12pt}\selectfont}
\newcommand{\erhao}{\fontline{21pt}\selectfont}
\usepackage{amsmath}
\usepackage{geometry}
\usepackage[]{lmodern}
\geometry{papersize={210mm,297mm}}
\geometry{left=3.18cm,right=3.18cm,top=2.54cm,bottom=2.54cm}
\CTEXsetup[indent={20pt},format={\heiti\zihao{3}\bfseries}]{section}
\CTEXsetup[indent={40pt},format={\heiti\zihao{3}\bfseries}]{subsection}
\CTEXsetup[indent={60pt},format={\heiti\zihao{-3}\bfseries}]{subsubsection}
\begin{document}
\begin{titlepage}
  \begin{center}
    \quad  \\
    {\CJKfontspec{STXingkai}  \zihao{2}北京航空航天大学}
  \end{center}
  \begin{center}
    {\CJKfontspec{LiSu}\huge 2021-2022 学年 第一学期}
    \vskip 2.5cm
  \end{center}
  \begin{center}
    {\CJKfontspec{LiSu}\zihao{1}《Python编程与智能车技术》}
    \newline
    {\CJKfontspec{LiSu}\zihao{-1}结课报告}
    \vskip 3.5cm
  \end{center}
  \begin{center}
    {\CJKfontspec{LiSu}\zihao{3} 班\quad 级\underline{周五8-10节}\qquad 学\quad 号\underline{21374230}\\}
    {\quad \\}
    {\CJKfontspec{LiSu}\zihao{3}姓\quad 名\underline{\quad 黄乙笑\quad}\qquad  成\quad 绩\underline{\qquad\qquad}}
    \vskip 2.0cm
  \end{center}
  {\CJKfontspec{LiSu}\zihao{3}所在小组组长:孔令琪\\}
  {\CJKfontspec{LiSu}\zihao{3}组\qquad\qquad 员:黄乙笑,杨辰润,罗俊午,范宏伟}
\end{titlepage}
  \section{Python语言知识体系}
  Python的强大之处,在于他灵活简单的语法、多样的数据类型、面向对象,还有各种强大的官方库或者第三方库。
  \subsection{Python的基础语法}
  与其他各种语言一样,python也有是以顺序、选择、循环为基础的,逻辑都差不多。
  \subsection{多样的数据类型}
  \subsubsection{一般的数据类型}
  Python也有整形、浮点型、布尔型、字符、字符串等,但python对整形和浮点型的限制没有c和c++这么严格,在某种意义上,可以把二者看作是一样的。
  \subsubsection{特殊的组合数据类型}
  列表list、元组tuple、集合set、字典dic,是python特有的几种组合数据类型,字符串也可以算入其中。通过这些数据类型,我们能较为方便地处理大量的数据。各个数据类型可以相互嵌套,组成更加复杂的数据类型。
  \par
  列表通过[ ]或者list来创建,类似于c语言中的数组。但列表可以进行索引、切片、遍历等操作,还能使用append、clear、copy、insert、pop、remove、reserve等方法进行复杂的操作。
  \par
  元组通过( )或者tuple创建,与列表类似,但他们之间最大的区别是:元组是不可以改变的,列表是能改变的。所以append、clear等方法都不适用于元组。
  \par
  集合通过\{ \}或set创建,与数学中的集合一样,集合是无序、排他的,可以增增删元素。集合有四种操作:交 \& ;并 \textbar ;差 - ;补 \^{} 。也有add、clear、len等方法。常用集合来进行去重。
  \par
  字典通过\{ \}或者dic创建,每一个元素是一个键值对,即一个键及其对应的值。字典也是无序、不可重复的。对于字典dict,我们可以用dict[键]来进行索引和添加新键。dict.get (key,default) 方法可以返回key所对应的值,如果键key不存在则返回default。
  \subsection{面向对象的编程}
  Python,Java,C++是面向对象的语言的代表。面向对象即“万物皆对象”,通过对象能实现各种操作。\par
  在python的面向对象编程中,主要有以下几个关键的概念:类class,类变量,局部变量,实例变量,对象,实例化,方法,继承,方法重写。\par
  类变量是在类中共有的变量,类中所有函数均可访问。局部变量是类中某个函数的变量,其他函数不能访问。实例变量以“self.变量名”来进行定义,之作用于调用此方法的对象,只能通过对象名访问,而不能通过类名访问。\par
  对象是通过类定义的数据结构实例,包括两个数据成员(类变量和实例变量)和方法。实例化是创建一个类的实例,类的具体对象。\par
  方法是类中定义的函数。继承是从父类中继承字段和方法,而方法重写是当父类中的方法无法满足需求时,对其进行改写。
  \subsection{第三方库}
  除了python自带的math、random、turtle等库,全球的开发者还共同开发了很多第三方库,我们一般需要通过pip来安装。\par 
  在我们的学习中,就用到了jieba、worldcloud、numpy、pandas、matploit、pillow、sklearn等库,为我们做项目提供了极大的方便。这也是python受到大家喜爱的原因之一吧。
  \newpage
  \section{机器学习}
  机器学习从上世纪中叶开始,经过了几十年的发展,已经较为成熟。机器学习可以粗略分为有监督学习、无监督学习、半监督学习和强化学习。其中,较为基础的是有监督学习。\par
  有监督学习是既有特征,又有标签的学习。常见的有神经网络、线性回归、逻辑回归、支持向量机(SVM)、朴素贝叶斯、决策树等,可分为分类和回归两大类。特征是输入的变量,一般用向量X表示,标签就是要输出的结果,一般应用向量Y表示。在训练集中,每个训练样本都有X和Y。\par
  无监督学习与有监督学习相比,没有标签。在实际应用中,无监督学习比有监督学习应用范围更广,但技术上更加困难。无监督学习大多应用在聚类和k-means中。\par 
  在python中,常用的机器学习库有sklearn等。可以实现数据预处理、数据集、特征选择、特征降维、模型构建(分类、回归、聚类)、模型评估(分类、回归、聚类)。值得一提的,sklearn自带很丰富的数据库,有手写数字数据集、波士顿房价数据集,还有乳腺癌和糖尿病数据集等。\par 
  功能更加强大的还有Tensoflow,Pytorch,Paddlepaddle等开源的机器学习库,通过这些库,即使不太理解机器学习算法的实现原理,我们仍然能写出性能很好的机器学习程序。
  \newpage
  \section{车辆识别项目}
  \subsection{基于全连接神经网络的车辆识别}
  全连接神经网络,又叫深度神经网络(Deep Neural Networks),是最基础的神经网络之一。\par 
	正如其名字所描述的,DNN是一种全连接的前馈神经网络。\par 
	神经网络就是以我们的生物上的神经元为基础出发的。生物上的神经元,通过树突接受来自感应器或者上一神经元的信号,并通过轴突输出给效应器或者下一神经元;神经网络中的神经元,或者说一个节点,接收来自输入层或上层神经元的数据,并输出给输出层或者下层神经元。人脑有多层,神经网络也有多层。
	全连接,就是前一层的每个神经元与后一层的每个神经元都有连接。这使得全连接神经网络较为复杂,需要的数据多,计算量也很大,非线性的拟合能力较好。当我们的隐藏层较多时,才能称作全连接神经网络,否则只是浅层神经网络。
	研究单个的神经元,我们要研究它的输入、输出,更要研究从输入到输出的过程。简单而言,将所有输入按照一定的权重(weight)加和,再加上偏置(bias),经过激活函数之后,进行输出。如此完成一个从输入到输出的过程之后,模型会有一个输出值,与目标值之间有差距,我们就用损失函数来刻画这个差距。DNN的最终目标就是:通过改变权重、偏置,使得损失函数最小。
	上述过程的数学表达如下:
  $$
    z = a_1 \omega_1+a_2 \omega_2 +...+a_k\omega_k+b=A \times \Omega + b\\
  $$
  $$
    \omega \mbox{为权重,b为偏置。A为} a_1,a_2\cdots \mbox{组成的向量,}\Omega\mbox{为}\omega_1,\omega_2\cdots \mbox{组成的向量}\\
  $$
  $$
    y = \sigma(z),\sigma\mbox{为激活函数,}y\mbox{为输出}\\
  $$
  $$
    \mbox{改变}A,\Omega,\mbox{使得}loss(Y)\mbox{最小},Y\mbox{为}y\mbox{组成的向量}
  $$
	如何选取激活函数、损失函数,如何让机器改变参数使得损失函数最小,是我们需要思考的三个问题。
	常用的激活函数有四种:
$$
  Sigmoid\mbox{函数}: \frac{1}{1+e^{-x}} 
$$
$$
  Tanh\mbox{函数}: \frac{e^x-e^{-x}}{e^x+e^{-x}} 
$$
$$
  Softsign\mbox{函数}: \frac{x}{1+|x|} 
$$
$$
  ReLU\mbox{函数}:max(0,x)
$$
	如果不使用激活函数,那么输入与输出仍然会保持线性关系,无法拟合非线性关系。激活函数有各自的优缺点,如Sigmoid函数与Tanh函数,在自变量很大或者很小的时候会出现梯度消失的情况,即函数值随自变量变化不大。我们一般在输出层采用Softmax函数,即归一化指数函数,将每一个结果以其概率表现出来,总和为1。
	损失函数较常用的有平方差和交叉熵两种。
	平方差函数:$loss(Y)=\frac{1}{n} \sum_{i=0}^{n}{(y_i-y)^2}$,刻画输出与标准值的差距,常用于回归。
	交叉熵函数:$H(p,q)=-sum{p(x)\log⁡{q(x)}}$,其中q是经过Softmax激活之后的输出概率分布,p是实际的概率分布,x是每个概率分布对应的分量。它刻画预测的概率与实际的概率的差值,输出的常用于分类。如在智能车分类中,输入是一辆摩托车,的概率:摩托车0.7,汽车0.2,货车0.1。即$p=(1,0,0);  q=(0.7,0.2,0.1) ;\  H(p,q)=-(log⁡0.7+0+0)=-log⁡0.7$。
	如何改变参数?
	最常用的是基于梯度下降法的反向传播算法。即让参数值沿着函数的梯度的反方向下降。为减少计算量,一般采用随机梯度下降,使用一个batch的数据进行参数更新。
	为了防止过拟合,减小计算量,我们也可以使用drop out,在每一轮训练中随机忽略一些节点。
  \subsection{基于卷积神经网络的车辆识别}
  卷积神经网络(Convolutional Neural Networks, CNN)有三个特点:局部连接、权重共享、下采样,这写特性使得它摆脱了全连接神经网络的缺点:结构不灵活,参数太多,训练速度慢。\par 
  局部连接就是选取局部进行识别,关注某些特殊的区域。权重共享就是各个局部的神经元参数相同。下采样就是让尺寸变小,像素变少。三个方法一起在保证准确度的同时减少了卷积神经网络的计算量。\par 
  卷积神经网络一般的步骤是输入$\rightarrow$卷积层$\rightarrow$ 池化层 $ \rightarrow \cdots $卷积层$\rightarrow$ 池化层 $\rightarrow$ 全连接$\rightarrow$softmax$\rightarrow$输出\par
  卷积层是用卷积核对图像进行卷积,可以理解为:输入图像(input image) * 卷积核 (kernel) = 特征(feature map),\*是卷积操作,即矩阵之间的数乘。这是最基础的单核单通道卷积。一般的图像是三通道的,我们就需要把每个通道卷积之后的特征进行进行合并,此处每个通道的卷积核是不一样的,因为一个卷积核只能表示一个特征。还有多核卷积,即每个通道也采用不同的卷积核进行卷积,最终叠加。\par
  池化层(pooling)又叫做下采样层,用来压缩数据,缩小feature map尺度。常用最大池化(max pooling) 或者平均池化 (average pooling)两种池化层进行池化。\par 
  损失函数等其他参数与全连接大致相同。
  \newpage
  \section[]{对本课程的学习总结}
  \subsection{个人学习反思}
  在学习方法上,我并不认为这门课是一门一般的通识课,而是一门入门人工智能的课程。我在课上认真做笔记,重要的程序在课下要么自己打一遍,要么看把它看懂。但是由于课时安排节奏太快,课余事件也不是特别充裕,越到后面,特别是机器学习那一块,我就放低了对自己的要求。对sklearn和paddlepaddle库中的东西搞不太懂,调参数、加层数也不明白为什么要这样做,这样做有什么好处有什么坏处。\par 
  在小组作业中,我们组的选题是用pygame做一个游戏。有杨辰润学长搭建框架,写主要部分,我和罗俊午学长做细节。我主要负责的是写other\_thing.py,做出漂浮物和能量球。刚开始我真的是一头雾水,一是不知道我该写什么,二是不知道我该怎么写。需求不是很明确,只说了一个大概的方向;对pygame完全没有了解,面向对象也学的不是特别扎实。后来经过自己的思考、与学长的讨论、阅读写好的程序框架,对游戏的逻辑以及需要我去实现的功能有了大致的了解。然后再在百度的帮助下,跌跌撞撞写出了一个大概。但是程序逻辑很混乱,我又自己进行了一边code review,杨辰润学长把我们的程序整合到一起之后,又改了一些细节,才使得整个程序看起来很清爽。写代码其实是一门艺术。但是也有遗憾之处,本来我想的是这两个东西会动,但是我写着写着就把这事忘了,在我看最终版的 时候才发现speed\_x,speed\_y根本没用上,遂作罢。
  \subsection{对课程的总结和建议}
  在寒假选这门课的时候,就看到了学长学姐们对这门课非常一致的评价:“很硬,但值得一上。”当时仰仗着自己有一丢丢python基础,对python语言又比较感兴趣,就选了这门课。现在这门课结课了,如果要我给一个评价,我也会说那句话:“很硬,但值得一上。”不仅仅是在知识层面上得到了很大的提升,在个人学习能力上、小组合作能力上,我都感觉我有进步。\par 
  我是一个很爱接受新鲜事物、什么都想去尝试一下的人(这篇总结便是用才学的 latex写的)。我可能什么东西都知道一点,但不专精,在 pyhon这件事上也是如此。以后有了更多的空闲时间,我应该会更深入地学习 python,学习机器学习相关内容。\par 
  对于这门课的建议,我觉得是否可以向教务处申请延长课时、增加学分,使得我们有更多时间去学习机器学习相关内容,做更多项目。
\end{document}